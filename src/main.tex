\documentclass{article}
\usepackage{graphicx}
\usepackage[mathscr]{euscript}
\usepackage{parskip}
\usepackage{amsmath}
\usepackage{amsthm}
\usepackage{amssymb}
\setlength{\parindent}{0pt}

\title{Book of Proof Summary}
\author{Raphaël Hermans}
\date{February 2026}

\begin{document}

\maketitle

Contains important examples, theorems, proofs and definitions from the book, not every detail

\section{Sets}

\begin{itemize}
    \item $\mathscr{P}(A)$ is the powerset of A
    \item Russel's paradox: $X = \{A \text{ is a set } | A \in A\}$. Now $X \in X \iff X \notin X$
    \item Following the Zermelo–Fraenkel axioms avoids Russel's paradox
    \item $U$ is the universal set and is context dependent
    \item $\bar{A} = A^c$ is the complement of $A$, $U - A$
\end{itemize}

\section{Logic}

\begin{itemize}
    \item $\neg (\forall x, P(x)) \iff \exists x, \neg P(x)$
    \item $\neg (\exists x, P(x)) \iff \forall x, \neg P(x)$
    \item $(P \implies Q) \iff (\neg P \lor Q)$
    \item Exercise in English: "Whenever I have to choose between two evils, I
    choose the one I haven't tried yet." equates to "For all choices of two
    evils, if I have not tried an evil, I pick that evil.", which translates to
    \[\forall (e_1,e_2) \in E, \forall e \in (e_1,e_2), \neg \text{tried}(e)
    \implies \text{pick(e)}.\] Negating this gives \[ \neg (\forall (e_1,e_2)
    \in E, \forall e \in (e_1,e_2), \neg \text{tried}(e) \implies
    \text{pick(e)})\]
    \[\exists (e_1,e_2) \in E, \exists  e \in (e_1,e_2), \neg (\text{tried}(e) \lor \text{pick(e)})\]
    \[\exists (e_1,e_2) \in E, \exists e \in (e_1,e_2), \neg \text{tried}(e)
    \land \neg\text{pick(e)}\] Translating this back into English gives: "There
    exists a choice of two evils, for which there is an evil which I haven't
    tried and didn't pick."
\end{itemize}

\section{Counting}

\begin{itemize}
    \item Lists $(a,b,c, \dots)$ are ordered and can contain duplicates. Sets
    $\{a,b,c, \dots \}$ are unordered and do not contain duplicates
    \item $\binom{n}{k}$ is read as "$n$ choose $k$"
    \item Multisets are sets which can contain elements multiple times. They are
    denoted with $[]$ brackets
    \item The cardinality of a multiset $A$ is the number of elements in $A$
    including repetitions
    \item The number of $k$-element multisets that can be made from an n-element
    set is $\binom{k+n-1}{n-1} = \binom{k+n-1}{k}$
    \begin{proof}
         We can organize the elements of any multiset in alphabetical order, so
         that any multiset can be written as a sequence of $k$ characters $*$
         and $n-1$ characters $|$. In such a representation, the character $*$
         denotes the appearance of an element from the set and $|$ denotes a
         separator, i.e. a shift to the next element of the set. E.g. the
         multiset $[a,a,b,c]$ generated from $\{a,b,c\}$ can be written as the
         sequence $**|*|*$. Thus, there are as many multisets as there are
         configurations of the characters. And since there are exactly $k+n-1$
         spots to place the $n-1$ bars, the number of possible multisets is
         $\binom{k+n-1}{n-1}$.
    \end{proof}
    \item \textbf{Division Principle:} \textit{Suppose you divide $n$ objects
    into $k$ boxes. At least one box must contain at least $\lceil n/k\rceil$
    and one box must contain at most $\lfloor n/k \rfloor$ objects.}
    \begin{proof}
        We will prove that the negation of the division principle is false,
        proving that the principle is, in fact, true. Assume B the set of boxes
        and $|b|$ the number of elements in box $b$. Suppose now that the
        division principle does not hold. Then we know that
        \[\neg((\exists b \in B, |b| \ge \lceil n/k \rceil) \land(\exists b \in
        B, |b| \le \lfloor n/k \rfloor)) \]
        \[(\forall b \in B, |b| < \lceil n/k \rceil) \land(\forall b \in B, |b|
        > \lfloor n/k \rfloor) \]
        \[\forall b \in B, \lfloor n/k \rfloor < |b| < \lceil n / k \rceil.\]
        Now there are two possible cases: $\lfloor n/k \rfloor = \lceil n / k
        \rceil$ and $\lfloor n/k \rfloor = \lceil n / k \rceil - 1$. Both cases
        immediately lead to an impossible solution, since there is no valid
        number of elements $|b|$ to choose. Therefore, if the division principle
        does not hold, there is no way to distribute the elements in a valid
        way.
    \end{proof}
\end{itemize}

\end{document}
