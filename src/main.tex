\documentclass{article}
\usepackage{graphicx}
\usepackage[mathscr]{euscript}
\usepackage{parskip}
\usepackage{amsmath}
\usepackage{amsthm}
\usepackage{amssymb}
\usepackage{darkmode}
\setlength{\parindent}{0pt}

% Disable for actual printing
\enabledarkmode

\title{Book of Proof Summary}
\author{Raphaël Hermans}
\date{February 2026}

\begin{document}

\maketitle

% ========== PART 0 ========== %
Contains examples, theorems, proofs, exercises and definitions from the book
that I found important, not every detail (3rd edition)


% ========== PART 1 ========== %
\section{Fundamentals}

\subsection{Sets}

\begin{itemize}
    \item \textbf{Powerset:} $\mathscr{P}(A)$ is the powerset of $A$, i.e.
    the set of all subsets, sometimes denoted $2^A$, because for a set with
    cardinality $n$, there are $2^n$ subsets.
    \item \textbf{Russel's paradox:} The set $X = \{A \text{ is a set } | A \in
    A\}$ implies the logical inconsistency $X \in X \iff X \notin X$. Following
    the Zermelo–Fraenkel axioms avoids Russel's paradox
    \item \textbf{Set builder notation:} $\{x \in A | P(x)\}$ is the set of all elements
    $x$ in $A$ such that $P(x)$ is true
    \item \textbf{Universal set: }$U$ is generally considered to be the universal set and is context
    dependent. E.g. for $\{(x,y) \in \mathbb{R}^2 | x^2 + y^2 = 1\}$, $U =
    \mathbb{R}^2$
    \item \textbf{Complement: } $\bar{A} = A^c$ is the complement of $A$, $U - A$
\end{itemize}

\subsection{Logic}

\begin{itemize}
    \item $\neg (\forall x, P(x)) \iff \exists x, \neg P(x)$
    \item $\neg (\exists x, P(x)) \iff \forall x, \neg P(x)$
    \item $(P \implies Q) \iff (\neg P \lor Q)$
    \item Exercise in English (personal solution of 2.10 P12): "Whenever I have to choose between two evils, I
    choose the one I haven't tried yet." equates to "For all choices of two
    evils, if I have not tried an evil, I pick that evil.", which translates to
    \[\forall (e_1,e_2) \in E, \forall e \in (e_1,e_2), \neg \text{tried}(e)
    \implies \text{pick(e)}.\] Negating this gives \[ \neg (\forall (e_1,e_2)
    \in E, \forall e \in (e_1,e_2), \neg \text{tried}(e) \implies
    \text{pick(e)})\]
    \[\exists (e_1,e_2) \in E, \exists  e \in (e_1,e_2), \neg (\text{tried}(e) \lor \text{pick(e)})\]
    \[\exists (e_1,e_2) \in E, \exists e \in (e_1,e_2), \neg \text{tried}(e)
    \land \neg\text{pick(e)}\] Translating this back into English gives: "There
    exists a choice of two evils, for which there is an evil which I haven't
    tried and didn't pick."
\end{itemize}

\subsection{Counting}

\begin{itemize}
    \item \textbf{Lists} $(a,b,c, \dots)$ are ordered and can contain duplicates. Sets
    $\{a,b,c, \dots \}$ are unordered and do not contain duplicates
    \item $\binom{n}{k}$ is read as "$n$ choose $k$"
    \item \textbf{Multisets} are sets which can contain elements multiple times. They are
    denoted with $[]$ brackets
    \item The cardinality of a multiset $A$ is the number of elements in $A$
    including repetitions
    \item The number of $k$-element multisets that can be made from an n-element
    set is $\binom{k+n-1}{n-1} = \binom{k+n-1}{k}$
    \begin{proof}
         We can organize the elements of any multiset in alphabetical order, so
         that any multiset can be written as a sequence of $k$ characters $*$
         and $n-1$ characters $|$. In such a representation, the character $*$
         denotes the appearance of an element from the set and $|$ denotes a
         separator, i.e. a shift to the next element of the set. E.g. the
         multiset $[a,a,b,c]$ generated from $\{a,b,c\}$ can be written as the
         sequence $**|*|*$. Thus, there are as many multisets as there are
         configurations of the characters. And since there are exactly $k+n-1$
         spots to place the $n-1$ bars, the number of possible multisets is
         $\binom{k+n-1}{n-1}$.
    \end{proof}
    \item \textbf{Division Principle:} \textit{Suppose you divide $n$ objects
    into $k$ boxes. At least one box must contain at least $\lceil n/k\rceil$
    and one box must contain at most $\lfloor n/k \rfloor$ objects.}
    \begin{proof}
        We will prove that the negation of the division principle is false,
        proving that the principle is, in fact, true. Assume B the set of boxes
        and $|b|$ the number of elements in box $b$. Suppose now that the
        division principle does not hold. Then we know that
        \[\neg((\exists b \in B, |b| \ge \lceil n/k \rceil) \land(\exists b \in
        B, |b| \le \lfloor n/k \rfloor)) \]
        \[(\forall b \in B, |b| < \lceil n/k \rceil) \lor(\forall b \in B, |b| >
        \lfloor n/k \rfloor) \] Now we have an expression of the form $P \lor
        Q$, which is false if both $P$ and $Q$ are false. The first proposition
        must be false, since $k \lceil n/k \rceil \le n$, so if $|b| < \lceil
        n/k \rceil$ for all $b$, then the amount of objects in boxes is $k|b| <
        n$, i.e. not all the boxes are filled. A similar argument holds for the
        second proposition.
    \end{proof}
    \item \textbf{Pigeonhole Principle:} \textit{Suppose you divide $n$ objects
    into $k$ boxes. If $n>k$, then at least one box contains more than one object.
    If $n<k$, then at least one box is empty.}
    \item Example proof using the division principle (personal solution of 3.9
    P10): Given a sphere S, a great circle of S is the intersection of S with a
    plane through its center. Every great circle divides S into two parts. A
    hemisphere is the union of the great circle and one of these two parts. Show
    that if five points are placed arbitrarily on S, then there is a hemisphere
    that contains four of them.
    \begin{proof}
        Consider the first two points. They define a unique great circle. The
        remaining three points are to be placed in either one of the two
        hemispheres. Now, the according to the division pricniple, at least one
        hemisphere must contain at least $\lceil 3/2 \rceil = 2$ points. Thus,
        there is a hemisphere containing at least the first two points and two
        of the remaining three points, i.e. four points in total.
    \end{proof}
    \item \textbf{Combinatorial proof} is a method of proving two different
    expressions are equal by showing that they are both answers to the same
    counting question
    \item Example of combinatorial proof (personal solution of 3.10 P12): Show
    that $\sum_{k=1}^n \binom{n}{k} \binom{k}{m} = \binom{n}{m} 2^{n-m}$.
    \begin{proof}
        Assume we have a set of $n$ ordered elements. Then the right-hand side
        of the equation counts the number of ways in which we can choose a
        subset $A$ of size $m$ and then choose whether to include the
        remaining $n-m$ elements in a second set $B$. The left-hand side of the
        equation counts the same thing differently. First, we choose the
        number of elements $k$ to include in the total subset $A \cup B$. Now,
        for each possible $k$, there are $\binom{n}{k}$ ways to choose the $k$
        elements from the original set. From these $k$ elements, we need to
        choose $m$ elements to be in the first subset $A$. The remaining
        elements will be part of the second subset $B$. There are $\binom{k}{m}$
        ways to make this choice, so that the total number of ways to choose
        such a subset is $\sum_{k=1}^n \binom{n}{k} \binom{k}{m}$.
    \end{proof}
\end{itemize}


% ========== PART 2 ========== %
\section{How to Prove Conditional Statements}

\subsection{Direct Proof}

\begin{itemize}
    \item \textbf{Theorem:} A true, significant statement to be proved
    \item \textbf{Lemma:} An auxiliary theorem used to prove a larger theorem
    \item \textbf{Corollary:} A theorem that follows easily from another theorem
    \item \textbf{Proposition:} A true statement that is not as significant as a theorem
    \item \textbf{Parity:} Two integers have the same parity if they are both even or both odd,
    otherwise they have opposite parity
    \item \textbf{Composite:} A positive integer that is not prime
    \item If $a$ divides $b$ we write $a|b$, i.e. $\exists c \in \mathbb{Z}, b = ac$
    \item The greates common divisor of $a$ and $b$ is denoted $\gcd(a,b)$, the least
    common multiple is denoted $\text{lcm}(a,b)$
    \item \textbf{Division Algorithm:} $\forall a, b \in \mathbb{Z}, b > 0, \exists! q, r \in
    \mathbb{Z}, 0 \le r < b$  such that $a = bq + r$
    \item In direct proofs, it can be a good idea to work at the proof from both ends, from
    the assumptions down, and from the conclusion up, until both sides meet
    \item WLOG means "without loss of generality", and is generally used when a proof for
    one case can be easily adapted to the other cases
    \item Direct proof example (personal solution of 4 P28): Let $a,b,c \in
    \mathbb{Z}$. Suppose $a$ and $b$ are not both zero, and $c \ne 0$. Prove
    that $c \cdot \gcd(a,b) \le \gcd(ca,cb)$.
    \begin{proof}
        Let $d$ be the greatest common divisor of $a$ and $b$, so that $d_1 =
        \gcd(a,b)$. Then there exist integers $x_1$ and $y_1$ such that $d_1x_1
        = a$ and $d_1y_1 = b$. Now, consider $d_2 = \gcd(ca,cb)$. There exist
        integers $x_2$ and $y_2$ such that $d_2x_2 = ca$ and $d_2y_2 = cb$.
        Substituting $a$ and $b$ in these equations gives $ca = cd_1x_1$ and $cb
        = cd_1y_1$. Thus, $cd_1$ divides both $ca$ and $cb$. And since $d_2$ is
        the greatest common divisor of $ca$ and $cb$, we have that $cd_1 \le
        d_2$, or $c \cdot \gcd(a,b) \le \gcd(ca,cb)$.
    \end{proof}
\end{itemize}

\subsection{Contrapositive Proof}

\begin{itemize}
    \item \textbf{Contrapositive:} The contrapositive of the statement $P
    \implies Q$ is the logically equivalent statement $\neg Q \implies \neg P$
    \item It is said that $a$ and $b$ are "congruent modulo $n$" if $n | (a-b)$, otherwise
    stated as $a \equiv b \mod n$
    \item General guidelines for good proofs:
    \begin{itemize}
        \item Begin each sentence with a word, not a symbol, otherwise it might
        create ambiguity
        \item End each sentence with a period, even when the sentence ends in a
        mathematical expression.
        \item Always separate mathematical expressions with text to avoid
        ambiguity
        \item Use the first-person plural ("we") to make the reader feel
        included
        \item Each new symbol must be explained when it is first introduced
        \item Watch out for the words "it" and "this" as they can be ambiguous
        \item It is helpful to tip the reader off as to what type of proof you are
        using: direct, contrapositive, contradiction, induction, etc.
    \end{itemize}
    \item Example of contrapositive proof (personal solution of 5 P32): If $a
    \equiv b \mod n$, then $a$ and $b$ have the same remainder when divided by
    $n$.
    \begin{proof}
        Assume $a$ and $b$ have a different remainder when divided by $n$, such that
        \[a = nq_1 + r_1\]
        \[b = nq_2 + r_2\]
        with $0 \le r_1, r_2 < n$ and $r_1 \ne r_2$. Now, we can write that
        $a - b = n(q_1 - q_2) + (r_1 - r_2)$. Since $r_1 \ne r_2$, we know that
        $a - b$ does not divide $n$, so that $a \not\equiv b \mod n$. Thus, by
        contrapositive, if $a \equiv b \mod n$, then $a$ and $b$ have the same
        remainder when divided by $n$. 
    \end{proof}
\end{itemize}

\subsection{Proof by Contradiction}

\begin{itemize}
    \item In proof by contradiction, we assume $\neg P$ and logically derive a
    contradiction $C \land \neg C$ to conclude $P$. This is a vald proof, since
    $\neg P \implies C \land \neg C$ is logically equivalent to $P$.
    \begin{proof}
        \begin{align*}
        &\neg P \implies C \land \neg C\\
        &\iff P \lor (C \land \neg C)\\
        &\iff P \lor F\\
        &\iff P
        \end{align*}
    \end{proof}
    \item Example of proof by contradiction (personal solution of 6 P24): The number
    $\log_2 3$ is irrational.
    \begin{proof}
        Assume $\log_2 3$ is rational, so that $\log_2 3 = \frac{a}{b}$ with $a$
        and $b$ positive integers. Thus, $2^{\frac{a}{b}} = 3$, so that $2^a =
        3^b$. This is a contradiction since $2^a$ is even and $3^b$ is odd. The
        proposition that $2^a$ is even can easily be proven through mathematical
        induction. The same goes for the second proposition, that $3^b$ is odd.
        We will analyze only the second proposition. The base case is trivial:
        $3^1$ is an odd number. Now assume $3^{b-1}$ is odd, then $3^{b-1} = 2k
        + 1$ with $k \in \mathbb{Z}$, so that $3^b = 3 \cdot (2k + 1) = 6k + 3 =
        2 \cdot (3k + 1) + 1 = 2n + 1$ with $n \in \mathbb{Z}$.
    \end{proof}
\end{itemize}


% ========== PART 3 ========== %
\section{More on Proof}

\subsection{Proving Non-Conditional Statements}

\begin{itemize}
    \item In case of a biconditional theorem ($P \iff Q$), we need to prove both 
    $P \implies Q$ and $Q \implies P$. If we envision the implications as edges of
    a graph and the propositions as vertices, we are trying to make the graph fully
    connected.
    \item In case of multiple equivalent statements, $P \iff Q \iff R \iff \dots$
    we don't have to prove every possible conditional statement. Instead, we can
    prove the easiest conditional statements that make the graph fully connected.
    \begin{proof}
        A fully connected graph of implications implies that if any of the
        propositions are true, all of them must be true. This can be proven by
        Following the chain of implications through the graph. If it is truly
        fully connected, then the chain will reach every proposition,
        necessarily making all of them true.Likewise if one of them is false,
        they must all be false.
    \end{proof}
    \item In case of multiple equivalent statements, a circular pattern yields
    the fewest conditional statements that must be proved
    \item There are two types of existsence proofs: contstructive and
    non-contstructive proofs. Constructive proofs display an explicit example
    that proves the theorem, while non-constructive proofs prove an example
    exists without actually giving one.
\end{itemize}

\subsection{Proofs Involving Sets}

\begin{itemize}
    \item Proving $A \subseteq B$ is done by proving $a \in A \implies a \in B$
    \item Proving $A = B$ is done by proving $A \subseteq B \land B \subseteq A$
    \item There is an equivalence relationship between sets and propositional
    logic where sets correspond to propositions $P(x)$, operator $\cup$
    corresponds to $\lor$, operator $\cap$ to $\land$, and $\bar{\empty}$ to
    $\neg$.
    \item Equivalences that work for propositional logic also works for sets.
    E.g. DeMorgan's law, $\overline{A \cap B} = \overline{A} \cup \overline{B}$
    and $\overline{A \cup B} = \overline{A} \cap \overline{B}$. This also works
    for distributive laws, associate laws, etc.
    \item \textbf{Perfect number:} A number $p \in \mathbb{N}$ is perfect when
    it equals the sum of its positive divisors less than itself. E.g. $6 = 1 + 2
    + 3$.
    \item Example of a proof involving sets (personal solution of 8 P31).
    Suppose $B \ne \emptyset$ and $A \times B \subseteq C \times C$. Prove that
    $A \subseteq C$.
    \begin{proof}
        Since the order of the Cartesian products matters, we know that if $A
        \times B \subseteq B \times C$, it must also be true that $A \subseteq B$
        and $B \subseteq C$ (this can easily be proven through proof by
        contradiction). Thus, due to the transitive property of sets, $A
        \subseteq C$ is also true.
    \end{proof}
\end{itemize}

\subsection{Disproof}

\begin{itemize}
    \item \textbf{Conjecture:} A statement which we don't know to be true or false
    \item Disproving a statement $P$ boils down to proving $\neg P$. For that we can
    just use the aforementioned techniques.
    \item Example of disproof (personal solution to 9 P34). If $X \subseteq A
    \cup B$, then $X \subseteq A$ or $X \subseteq B$.
    \begin{proof}
        We will disprove this proposition through a counterexample. Suppose $A =
        \{a, c\}$ and $B = \{b, c\}$, then $A \cup B = \{a, b, c\}$. Now suppose
        that $X = \{a, b, c\}$. Then $X \subseteq A \cup B$, but not $X \subseteq
        A$ or $X \subseteq B$.
    \end{proof}
\end{itemize}

\subsection{Mathematical Induction}

\begin{itemize}
    \item Strong induction assumes $S_1, S_2, \dots, S_k$ to prove $S_{k+1}$,
    while weak induction only assumes $S_k$
    \item \textbf{Proof by Smallest Counterexample:} A hybrid of mathematical
    induction and proof by contradiction. It has four steps
    \begin{enumerate}
        \item Check that $S_1$ holds
        \item Suppose that not every $S_n$ holds
        \item Let $S_k$ be the first false statement
        \item Use the fact that $S_{k-1}$ is true and $S_k$ false to get a
        contradiction
    \end{enumerate}
    \item \textbf{Fundamental Theorem of Arithmetic:} \textit{Any integer $k$
    greater than 1 has a unique prime factorization $k = p_1 \cdot p_2 \cdot
    \dots \cdot p_n$ with $p_1, p_2, \dots, p_n$ primes}
\end{itemize}

% ========== PART 4 ========== %
\section{Relations, Functions, and Cardinality}

\subsection{Relations}

\begin{itemize}
    \item Any subset $R$ of a set $A \times A$ can be considered a relation
    \item We often abbreviate $(x,y) \in R$ as $xRy$
    \item A relation is \textbf{reflexive} if $\forall x \in A, xRx$
    \item A relation is \textbf{symmetric} if $\forall x,y \in A, xRy \implies yRx$
    \item A relation is \textbf{transitive} if $\forall x,y,z \in A ((xRy) \land (yRz))
    \implies xRz$
    \item \textbf{Equivalence relation:} A relation that is reflexive, symmetric and
    transitive
    \item \textbf{Equivalence class: } A subset of $A$ containing only element
    that are equal under a certain equivalence relation, usually denoted with a
    representative element $a$. Equivalence classes are usually written as $[a]
    = \{x \in A : xRa\}$
    \item Equivalence relations arise in many fields of mathematics. They
    are often hidden under the surface. E.g. the set of antiderivatives $F(x) + c$
    are considered equal
    \item Partitions and equivalence relations have a one-to-one correspondence.
    \item Relations can also arise from two different sets, so that $R \subseteq
    A \times B$
    \item Proving anything related to relations boils down to applying
    the definitions of reflexive, symmetric and transitive relations along with the
    aforementioned methods of proof
\end{itemize}

\subsection{Functions}

\begin{itemize}
    \item A function $f$ from $A$ to $B$ (denoted $f: A \rightarrow B$) is a relation
    $f \subseteq A \times B$ that contains exactly one pair of the form $(a,b)$ for 
    every $a$ in $A$ (denoted $f(a) = b$).
    \item A function $f: A \rightarrow B$ contains a pair $(a,b)$ for every value $a$
    in its domain $A$, although it must not have a pair for every value $b$ in its
    codomain $B$
    \item The range is the subset $\{f(a) : a\in A\}$ of the codomain $B$
    \item Since a function is really just a relation and a relation is really
    just a set, a function is also just a set, in theory
    \item We consider functions to be equal ($f = g$) when their set representations
    are equal, i.e. when $f(x) = g(x)$ for every $x \in A$
    \item \textbf{Injection:} $\forall a, a' \in A, a \ne a' \implies f(a) \ne
    f(a')$ (one-to-one)
    \item \textbf{Surjection:} $\forall b \in B, \exists a \in A, f(a) = b$
    (onto)
    \item \textbf{Bijection:} injection and surjection
    \item Proving anything about function, like proving anything about relations
    boils down to applying the definitions of injections, surjections,
    bijections and the aforementioned methods of proof
    \item \textbf{Pigeonhole Principle:} (applied to functions) For a function
    $f: A \rightarrow B$:
    \begin{enumerate}
        \item If $|A| > |B|$, then $f$ is not injective
        \item If $|A| < |B|$, then $f$ is not surjective
    \end{enumerate}
    \item The inverse relation $R^{-1}$ is defined as $\{(y,x) : (x,y) \in R\}$
    \item The inverse function can only be defined when $f$ is bijective, in
    which case $f^{-1}$ is defined such that $f \circ f^{-1} = i_B$ and $f^{-1}
    \circ f = i_A$ 
    \item \textbf{Image:} $f(X) = \{f(x) : x \in X\}$
    \item \textbf{Preimage:} $f^{-1}(X) = \{f^{-1}(x) : x \in X\}$
    \item Example proof using functions (personal solution 12 P14): Let $f: A
    \rightarrow B$ be a function, and $Y \subseteq B$. Prove or disprove:
    $f^{-1}(f(f^{-1}(Y))) = f^{-1}(Y)$.
    \begin{proof}
        This statement is obviously true, but to prove it, we need to revisit
        definitions. $f^{-1}(f(f^{-1}(Y)))$ is defined as $\{f^{-1}(x) : x \in
        C\}$ with $C = f(f^{-1}(Y))$, so if we can prove that $C = Y$, then we
        are done. $C$ is defined as 
        \begin{align*}
        C &= \{f(x) : x \in f^{-1}(Y)\} \\
          &= \{f(x) : x \in \{f^{-1}(x') : x' \in Y\}\} \\
          &= \{f(f^{-1}(x')) : x' \in Y\} \\
          &= \{i_B(x') : x' \in Y\} \\
          &= Y
        \end{align*}
    \end{proof}
\end{itemize}

\subsection{Proofs in Calculus}
\begin{itemize}
    \item \textbf{Limit: } $\lim_{x \rightarrow c}f(x) = L$ if $\forall
    \epsilon > 0, \exists \delta > 0, (0 < |x - c| < \delta) \implies (|f(x) -
    L| < \epsilon)$
    \item The rest of this chapter mostly recaps high-school level calculus, not
    interesting
\end{itemize}

\subsection{Cardinality of Sets}

\begin{itemize}
    \item Two sets are said to have the same cardinality, written $|A| = |B|$,
    if there exists a bijection $f: A \rightarrow B$
    \item Any set with equal cardinality to $|\mathbb{N}|$ is considered
    countably infinite. E.g. $|\mathbb{N}| = |\mathbb{Z}|$.
    \begin{proof}
        Consider the bijection $f: \mathbb{N} \rightarrow \mathbb{Z}$ where
        $f(n) = n/2$ if $n$ even and $f(n) = -\frac{n-1}{2}$ if $n$ uneven $f:
        \{1,2,3,4,5,\dots\} \mapsto \{0,1,-1,2,-2,\dots\}$. To prove this is a
        bijection, we must prove it is both an injection and a surjection.
        Firstly, it is a surjection, since $\forall z \in \mathbb{Z}, \exists n
        \in \mathbb{N}, f(n) = z$, namely $n = 2z$ if $z > 0$ and $n = -2z + 1$
        if $z \le 0$. Secondly, it is an injection, since $n \ne
        n' \implies f(n) \ne f(n')$.
    \end{proof}
    \item It was George Cantor who first recognized that $|\mathbb{N}| \ne
    |\mathbb{R}|$ by proving that there are no surjective functions $f:
    \mathbb{N} \rightarrow \mathbb{R}$ (diagonal in the table argument). It is
    said that $\mathbb{R}$ is uncountably infinite
    \item Intervals in $\mathbb{R}$ are generally all uncountably infinite,
    since there are an infinite number of bijective function that can take you
    from any continuous interval to any other continuous interval in
    $\mathbb{R}$
    \item A set is considered countable if it is finite
    \item We use the first Hebrew letter "alpeh" to denote cardinalities. We
    write $\aleph_0 = |\mathbb{N}|$ and $\aleph_1 = |\mathbb{R}|$
    \item A set $A$ is countably infinite if we can arrange its elements in an
    infinite list $a_1, a_2, a_3, ...$, since $f: \mathbb{N} \rightarrow A : f(n)
    \mapsto a_n$ is a bijection
    \item The set of rational numbers $\mathbb{Q}$ is countably infinite
    \begin{proof}
        Every element in $\mathbb{Q}$ can be denoted as $\frac{a}{b}$ with $a$
        and $b$ integers with $b > 0$. If we place the values for $a$
        in the columns of a table and the values $b$ in the rows of a table, we
        can construct an infinite list that covers the whole table by drawing a
        line through the diagonals of the table.
    \end{proof}
    \item The Cartesian product of two countably infinite sets is also countably
    infinite. The proof is analogous to that of $|\mathbb{Q}| = \aleph_0$. Therefor
    $A^n$ must also be countably infinite if $A$ is countably infinite
    \item We say that $|A| < |B|$ when there is an injective function $f: A
    \rightarrow B$, but no bijective function
    \item We can prove that $\aleph_0 < \aleph_1$ with a similar argument as the
    one that is used to prove that $|\mathbb{R}| \ne |\mathbb{N}|$ (digits on
    the diagonal argument)
    \item If $A$ is any set, then $|A| < |\mathscr{P}(A)|$
    \begin{proof}
        To prove this theorem, we must prove that there is an injection $f: A
        \rightarrow \mathscr{P}(A)$, but no bijection. Proving there is an
        injection is easy, take $f(x) = \{x\}$ for example. Proving there is no
        bijection can be done by proving there is no surjection. Suppose
        \[B = \{x \in A : x \notin f(x)\} \subseteq A.\] This set has the
        property $B \in \mathscr{P}(A)$, but also $f(a) \ne B$ for all $a \in
        A$. Since this set exists for any function $f$, no $f$ can be
        surjective. To prove that $f(a) \ne B$ for all $a \in A$, consider the
        following cases. \\
        \underline{Case 1}: If $a \notin f(a)$ then $a \in B$. Consequently,
        $f(a) = B$ is impossible, since it would imply that $a \in B$. \\
        \underline{Case 2}: If $a \in f(a)$ then $a \notin B$. Consequently,
        $f(a) = B$ is impossibly, since it would imply that $a \notin B$. \\
        Simply put, no element $a$ can be mapped onto $B$, because if it were,
        then $B$ could not contain $a$, but if it does not contain $a$ it must
        contain $a$, leading to a contradiction.
    \end{proof}
    \item \textbf{Cantor-Bernstein-Schröder theorem:} To prove that $|A| = |B|$
    it suffices to prove that there exists and injection from $A$ to $B$ and from
    $B$ to $A$
    \item The Cantor-Bernstein-Schröder theorem can be used to prove that
    $|\mathscr{P}(\mathbb{N})| = |\mathbb{R}|$
    \item \textbf{Continuum hypothesis:} There exists no set with cardinality
    between that of $\mathbb{N}$ and that of its powerset
    \item It has been proven that the continuum hypothesis cannot be proven true
    or false. Therefor, accepting it as true leads to one version of set theory,
    while accepting it as false leads to another, both different, but valid
    versions
\end{itemize}

\end{document}
